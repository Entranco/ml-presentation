%{\setbeamercolor{background canvas}{bg=felix_bg}
\begin{frame}[t]
	\frametitle{Outline}
	\begin{enumerate}
		\item Introduction to evolving graphs
		\item Introduction to the inverse problem\footnote{Grindrod Peter and Higham Desmond J. 2010, Evolving graphs: dynamical models, inverse problems and propagation}
		\item My approach and results
	\end{enumerate}

	
\end{frame}


\begin{frame}[t]
	\frametitle{Graphs}
    \begin{center}
		\resizebox{\textwidth}{!}{
			\begin{tikzpicture}
    \tikzstyle{node} = [circle, fill=lightgray!90!black, draw, thick]
    \tikzstyle{edge} = [thick]
    \tikzstyle{edit} = [fill=editcol]
    \tikzstyle{lift} = [fill=liftcol]

    \node (1) [node] {};
    \node (2) [node, above right=0.5cm and 0.3cm of 1] {};
    \node (3) [node, right=0.3cm of 2] {};
    \node (4) [node, above left=0.5cm and 0.3cm of 1] {};
    \node (5) [node, below=0.5cm of 2] {};

    \draw (1) edge [edge] (2);
    \draw (2) edge [edge] (3);
    \draw (1) edge [edge] (4);
    \draw (2) edge [edge] (4);
    \draw (2) edge [edge] (5);

    \onslide<2-4>{
        \node (1) [node, fill=editcol] {};
    }

    \onslide<3-4>{
        \draw (1) edge [edge, color=liftcol] (2);
    }


\end{tikzpicture}
		}
	\end{center}

	\onslide<2-4>{
		{\large \edittext{Vertex: Someone/something with connections.}}
		
	}
	\vspace{.7em}

	\onslide<3-4>{
		{\large \lifttext{Edge: A connection between two vertices.}}
	}
	\vspace{.7em}

	\onslide<4>{
		Examples: Social media, airport network, the internet
	}

\end{frame}

\begin{frame}[t]
	\frametitle{Evolving Graphs}

	\begin{center}
		\resizebox{\textwidth}{!}{
			\begin{tikzpicture}
    \tikzstyle{node} = [circle, fill=lightgray!90!black, draw, thick]
    \tikzstyle{edge} = [thick]
    \tikzstyle{edit} = [fill=editcol]
    \tikzstyle{lift} = [fill=liftcol]

    \node (1) [node] {};
    \node (2) [node, above right=0.5cm and 0.3cm of 1] {};
    \node (3) [node, right=0.3cm of 2] {};
    \node (4) [node, above left=0.5cm and 0.3cm of 1] {};
    \node (5) [node, below=0.5cm of 2] {};

    \draw (1) edge [edge] (2);
    \draw (2) edge [edge] (3);
    \draw (2) edge [edge] (4);

    % First round of animation
    \onslide<1-2>{
        \draw (1) edge [edge] (4);
        \draw (2) edge [edge] (5);
    }

    \onslide<2-3>{
        \draw (3) edge [edge, dashed, color=addcol] (5) {};
    }

    \onslide<3>{
        \draw (1) edge [edge, color=removecol] (4);
        \draw (2) edge [edge, color=removecol] (5);
    }

    \onslide<4-5>{
        \draw (3) edge [edge] (5) {};
    }


\end{tikzpicture}
		}
	\end{center}

	\onslide<5>{
		We assign each edge two probabilities:
		\vspace{1.0em}

		\addtext{Birth Rate:} $\hat{\alpha}(e)$, the probability that the edge will appear.
		\vspace{1.0em}

		\removetext{Death Rate:} $\hat{\omega}(e)$, the probability that the edge will disappear.
	}
\end{frame}

\begin{frame}[t]
	\frametitle{Range-Dependent Graphs}

	Suppose some assignment of the integers ${1...n}$\\
	%\vspace{1.0em}
	$i_v$ = the integer assigned to vertex $v$
	
	\onslide<2-4>{
		\vspace{1.0em}
		Range: An edge (u, v) has range $|i_u - i_v|$
		\vspace{1.0em}
	}

	\onslide<3-4>{
		We can define birth rates and death rates as a function of range:
		\vspace{0.5em}
		$\hat{\alpha}(e) = \alpha(|i_u - i_v|)$\\
		$\hat{\omega}(e) = \omega(|i_u - i_v|)$
	}

	\begin{center}
		\resizebox{\textwidth}{!}{
			\begin{tikzpicture}
    \tikzstyle{node} = [circle, fill=lightgray!90!black, draw, thick]
    \tikzstyle{edge} = [thick]
    \tikzstyle{edit} = [fill=editcol]
    \tikzstyle{lift} = [fill=liftcol]

    \onslide<4>{
        \node (1) [node] {1};
        \node (2) [node, right=0.3cm of 1] {2};
        \node (3) [node, right=0.3cm of 2] {3};
        \node (4) [node, right=0.3cm of 3] {4};

        \draw (1) edge [edge] (2);
        \draw (2) edge [edge] (3);
        \draw (3) edge [edge] (4);
    }

\end{tikzpicture}
		}
	\end{center}

	\onslide<4>{
		Range of (v1, v4) = $|1-4| = 3$\\
		Range of (v2, v3) = $|2-3| = 1$
	}

\end{frame}